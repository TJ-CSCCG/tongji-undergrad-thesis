%!TEX program = xelatex
%!TEX encoding = UTF-8

\documentclass[oneside]{tongjithesis}

% Set thesis information
\school{计算机科学与技术学院}
\major{计算机科学与技术}
\student{1234567}{张三}
\thesistitle{基于机器学习的图像分类研究}{——以交通标志识别为例}
\thesistitleeng{Research on Image Classification Based on Machine Learning}{——Take Traffic Sign Recognition as an Example}
\thesisadvisor{李四 教授}
\thesisdate{2025}{4}{20}

\begin{document}

% Generate cover
\MakeCover

% Abstract pages
\pagestyle{firststyle}
\MakeAbstract{
    本文研究了基于机器学习的图像分类技术,并以交通标志识别为应用场景进行了实验验证。
    图像分类是计算机视觉中的基础任务,具有广泛的应用前景。本文首先综述了图像分类领域的经典算法和最新进展,
    然后提出了一种改进的卷积神经网络模型用于交通标志识别。实验结果表明,所提出的方法在准确率和计算效率方面
    均取得了良好的性能。
}{机器学习; 图像分类; 卷积神经网络; 交通标志识别}

% English Abstract
\MakeAbstractEng{
    This paper investigates image classification techniques based on machine learning, 
    with traffic sign recognition as an application scenario for experimental validation. 
    Image classification is a fundamental task in computer vision with extensive application prospects. 
    This paper first reviews classic algorithms and recent advances in the field of image classification, 
    then proposes an improved convolutional neural network model for traffic sign recognition. 
    Experimental results demonstrate that the proposed method achieves good performance in terms of 
    accuracy and computational efficiency.
}{Machine Learning; Image Classification; Convolutional Neural Network; Traffic Sign Recognition}

% Table of contents
\clearpage
\tableofcontents
\cleardoublepage

% Main content with page style
\pagestyle{mainstyle}

\section{引言}

图像分类是计算机视觉中的基础任务之一,其目标是将图像分配到预定义的类别中。
随着深度学习技术的发展,图像分类的性能得到了显著提升。
本文研究基于机器学习的图像分类技术,并以交通标志识别为例进行应用研究。

\subsection{研究背景}

交通标志识别是自动驾驶系统的重要组成部分,它能够帮助车辆理解道路环境,遵守交通规则。
准确、实时的交通标志识别对于提高自动驾驶安全性具有重要意义。

% ... More sections would go here ...

\section{结论}

本文提出了一种高效的交通标志识别方法,在准确率和计算效率方面均表现出色。
未来工作将探索模型的进一步轻量化,以及在嵌入式设备上的部署方案。

\end{document}
