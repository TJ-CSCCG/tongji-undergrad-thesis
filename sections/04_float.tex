\section{浮动体}\label{sec:float}

在本节(\cref{sec:float})中,我们将介绍浮动体的概念、分类及其在论文中的使用。浮动体包括图片、表格和算法等,它们的共同特点是可以自由浮动在页面中,不会影响正文的排版。
浮动体可以通过插入代码或使用专门的宏包来实现,常用的宏包有 \pkg{graphicx}、\pkg{float} 和 \pkg{algorithm} 等。在插入浮动体时,需要注意设置浮动体的位置和大小,以保证排版效果。
浮动体的正确使用可以提高论文的可读性和美观程度,但也需要注意不要过度使用或使其影响正文的连贯性。

\subsection{插图}

在 \TeX{} 中,插图功能是通过特定编译程序提供的机制实现的。不同的编译程序支持不同的图形格式。通常,图形都放置在浮动环境中。浮动的意思是,最终排版的图形位置不一定与源文件中的位置对应,这也是新手在使用 \LaTeX{} 时可能会遇到的问题。如果需要强制固定浮动图形的位置,请使用 \pkg{float} 宏包,并使用 \texttt{[H]} 参数。

\subsubsection{单个图形}

要插入单个图形,可以使用 \verb|\includegraphics| 命令,并在其参数中指定图像文件名和一些选项。例如:

\begin{figure}[!htb]
  \centering
  \includegraphics[width=0.5\textwidth]{example-image}
  \caption{这是一个示例图形}
  \label{fig:example}
\end{figure}

这段代码会在浮动体中插入一个名为 \verb|example-image| 的示例图形,并将其居中显示。图形下方会有一个标题“这是一个示例图形”,同时为该图形标上标签,以便于交叉引用。

需要注意的是,在使用 \verb|\includegraphics| 命令时,应该将图像文件放在与主文档相同的目录中,或者在参数中指定文件的完整路径。

\subsubsection{多个图形}

如果需要在同一浮动体中插入多个图形,可以使用 \verb|\subfigure| 命令或 \pkg{subcaption} 宏包。这些方法可以按照一定的排列方式组合多个图形,并分别为每个子图形添加标题和标签。
例如,可以使用 \verb|\subfigure| 命令插入两个水平并列的子图,如\cref{fig:example-subfig-1} 所示。这两个子图共用一个图形计数器,没有各自的子图题。

\begin{figure}[!htb]
  \centering
  \includegraphics[width=0.4\textwidth]{example-image-a}
  \hspace{1cm}
  \includegraphics[width=0.4\textwidth]{example-image-b}
  \caption{这是一个示例图形组}
  \label{fig:example-subfig-1}
\end{figure}

如果多个图形相互独立,不共用一个图形计数器,可以使用 \texttt{minipage} 或 \texttt{parbox},如\cref{fig:parallel1,fig:parallel2}。

\begin{figure}[!htb]
  \begin{minipage}{0.4\textwidth}
    \centering
    \includegraphics[width=\textwidth]{example-image-a}
    \caption{并排第一个图}
    \label{fig:parallel1}
  \end{minipage}\hfill
  \begin{minipage}{0.4\textwidth}
    \centering
    \includegraphics[width=\textwidth]{example-image-b}
    \caption{并排第二个图}
    \label{fig:parallel2}
  \end{minipage}
\end{figure}

为共用一个计数器的多个子图添加子图题,建议使用较新的 \pkg{subcaption} 宏包,不建议使用 \pkg{subfigure} 或 \pkg{subfig} 等宏包。

\pkg{subcaption} 宏包提供了 \cs{subcaptionbox} 命令,可以方便地实现子图的并排排列,并为每个子图添加子图题,如\cref{fig:subcaptionbox} 所示。此外,也可以使用 \pkg{subcaption} 宏包的 \cs{subcaption} 命令,将子图题放在子图之下,子图号用 a)、b) 等表示。可以将多个子图放在 \texttt{minipage} 中,用法与 \cs{caption} 相同。

\begin{figure}[!htb]
  \centering
  \subcaptionbox{左图}{\includegraphics[width=0.4\textwidth]{example-image-a}}
  \hspace{1cm}
  \subcaptionbox{右图}{\includegraphics[width=0.3\textwidth]{example-image-b}}
  \caption{包含子图题的示例(使用 subcaptionbox)}
  \label{fig:subcaptionbox}
\end{figure}

另外,\pkg{subcaption} 宏包还提供了 \pkg{subfigure} 和 \pkg{subtable} 环境,如\cref{fig:subfigure} 所示。这些环境的使用方法与 \pkg{subcaptionbox} 相似,可以实现子图的并排排列,并为每个子图添加子图题。

\begin{figure}[!htb]
  \centering
  \begin{subfigure}{0.3\textwidth}
    \centering
    \includegraphics[width=\textwidth]{example-image-a}
    \caption{左图}
  \end{subfigure}
  \hspace{1cm}
  \begin{subfigure}{0.4\textwidth}
    \centering
    \includegraphics[width=\textwidth]{example-image-b}
    \caption{右图}
  \end{subfigure}
  \caption{包含子图题的范例(使用 subfigure)}
  \label{fig:subfigure}
\end{figure}



总之,在使用 \LaTeX{} 插图时,应该注意浮动体的特性,尽量使用 \pkg{float} 宏包进行浮动体的控制,避免出现排版问题。同时,也应该熟悉插图命令的使用方法,以便于插入单个或多个图形,并设置适当的标题和标签。


\subsection{表格}

\subsubsection{基本表格}

编排表格时应简单明了、表达一致,内容明晰易懂,表文呼应,内容一致。表题应置于表格上方。

表格的编排建议采用国际通行的三线表\footnote{三线表以其形式简洁、功能分明、阅读方便而在科技论文中被推荐使用。三线表通常只有3条线,即顶线、底线和栏目线,没有竖线。}。可以使用 \pkg{booktabs} 宏包提供的 \cs{toprule}、\cs{midrule} 和 \cs{bottomrule} 命令来绘制三线表。同时,\pkg{longtable} 宏包可以与三线表很好地配合使用,可以排版较长的表格。
要编写三线表,需要在表格的头部、底部和每一栏之间绘制水平线。其中,\cs{toprule} 绘制顶部线,\cs{midrule} 绘制中部线,\cs{bottomrule} 绘制底部线。可以按照如下方式绘制三线表:

\begin{table}[!hpt]
  \caption{一个三线表}
  \label{tab:firstone}
  \centering
  \begin{tabular}{@{}llr@{}} \toprule
    \multicolumn{2}{c}{\textbf{Item}}                            \\ \cmidrule(r){1-2}
    \textbf{Animal} & \textbf{Description} & \textbf{Price} (\$) \\ \midrule
    Gnat            & per gram             & 13.65               \\
                    & each                 & 0.01                \\
    Gnu             & stuffed              & 92.50               \\
    Emu             & stuffed              & 33.33               \\
    Armadillo       & frozen               & 8.99                \\ \bottomrule
  \end{tabular}
\end{table}

\subsubsection{复杂表格}

我们经常会在表格下方标注数据来源,或者对表格里面的条目进行解释。可以用
\pkg{threeparttable} 实现带有脚注的表格,如\cref{tab:bigtable}。

\begin{table}[!htb]
  \centering
  \begin{threeparttable}[b]
    \caption{一个更大的三线表}
    \label{tab:bigtable}
    \begin{tabular}{@{}llrrrrrrr@{}} \toprule
      \multicolumn{2}{c}{\textbf{Item}} & \multicolumn{2}{c}{\textbf{Category 1}} & \multicolumn{2}{c}{\textbf{Category 2}} & \multicolumn{2}{c}{\textbf{Category 3}} & \multicolumn{1}{c}{\textbf{Total}}                                                                                     \\ \cmidrule(r){1-2} \cmidrule(lr){3-4} \cmidrule(lr){5-6} \cmidrule(lr){7-8} \cmidrule(l){9-9}
      \textbf{Animal}                   & \textbf{Description}                    & \textbf{Price} (\$)                     & \textbf{Quantity}                       & \textbf{Price} (\$)                & \textbf{Quantity} & \textbf{Price} (\$) & \textbf{Quantity} & \textbf{Price} (\$) \\ \midrule
      Gnat                              & per gram\tnote{a}                       & 13.65                                   & 100                                     & 12.35                              & 200               & 11.55               & 150               & 3650.00             \\
                                        & each                                    & 0.01                                    & 5000                                    & 0.01                               & 10000             & 0.009               & 20000             & 550.00              \\
      Gnu                               & stuffed                                 & 92.50                                   & 10                                      & 94.50                              & 15                & 96.50               & 20                & 5815.00             \\
      Emu                               & stuffed                                 & 33.33                                   & 25                                      & 34.33                              & 30                & 35.33               & 35                & 2704.95             \\
      Armadillo                         & frozen                                  & 8.99                                    & 50                                      & 7.99                               & 40                & 6.99                & 30\tnote{b}       & 1094.50             \\ \bottomrule
    \end{tabular}
    \begin{tablenotes}
      \item [a] 第一条脚注。
      \item [b] 第二条脚注。
    \end{tablenotes}
  \end{threeparttable}
\end{table}

为了编写更复杂的表格,我们可以使用 \href{https://www.tablesgenerator.com/}{Tables Generator} 来生成 \LaTeX{} 代码。该工具可以方便地设置表格的列数、行数、内容、格式、颜色等属性,支持常用的表格类型和样式,并且可以实时生成 \LaTeX{} 代码和预览效果,避免手动编写 \LaTeX{} 代码时出错。


如某个表需要转页接排,可以用 \pkg{longtable} 实现。接排时表题省略,表头应重复书
写,并在右上方写“续表 X”,如\cref{tab:performance}。

\begin{ThreePartTable}
  \begin{TableNotes}
    \item[a] 一个脚注
    \item[b] 另一个脚注
  \end{TableNotes}
  \begin{longtable}[c]{c*{6}{r}}
    \caption{实验数据}
    \label{tab:performance}                                                                                                                      \\
    \toprule
    \textbf{测试程序\tnote{b}} & \multicolumn{1}{c}{\textbf{正常运行}}   & \multicolumn{1}{c}{\textbf{同步}}
                           & \multicolumn{1}{c}{\textbf{检查点}}    & \multicolumn{1}{c}{\textbf{卷回恢复}}
                           & \multicolumn{1}{c}{\textbf{进程迁移}}   & \multicolumn{1}{c}{\textbf{检查点}}                                              \\
                           & \multicolumn{1}{c}{\textbf{时间} (s)} & \multicolumn{1}{c}{\textbf{时间} (s)}
                           & \multicolumn{1}{c}{\textbf{时间} (s)} & \multicolumn{1}{c}{\textbf{时间} (s)}
                           & \multicolumn{1}{c}{\textbf{时间} (s)} & \textbf{文件}(KB)                                                               \\
    \midrule
    \endfirsthead
    \multicolumn{7}{r}{\textbf{续表~\thetable}}                                                                                                    \\

    \toprule
    \textbf{测试程序\tnote{b}} & \multicolumn{1}{c}{\textbf{正常运行}}   & \multicolumn{1}{c}{\textbf{同步}}
                           & \multicolumn{1}{c}{\textbf{检查点}}    & \multicolumn{1}{c}{\textbf{卷回恢复}}
                           & \multicolumn{1}{c}{\textbf{进程迁移}}   & \multicolumn{1}{c}{\textbf{检查点}}                                              \\
                           & \multicolumn{1}{c}{\textbf{时间} (s)} & \multicolumn{1}{c}{\textbf{时间} (s)}
                           & \multicolumn{1}{c}{\textbf{时间} (s)} & \multicolumn{1}{c}{\textbf{时间} (s)}
                           & \multicolumn{1}{c}{\textbf{时间} (s)} & \textbf{文件}(KB)                                                               \\
    \midrule
    \endhead
    \hline
    \multicolumn{7}{r}{续下页}
    \endfoot
    \insertTableNotes
    \endlastfoot
    CG.C.2                 & 23.05                               & 0.002                               & 0.116          & 0.035 & 0.589 & 32491  \\
    CG.A.4                 & 15.06                               & 0.003                               & 0.067          & 0.021 & 0.351 & 18211  \\
    CG.A.8                 & 13.38                               & 0.004                               & 0.072          & 0.023 & 0.210 & 9890   \\
    CG.B.2                 & 867.45                              & 0.002                               & 0.864          & 0.232 & 3.256 & 228562 \\
    CG.B.4                 & 501.61                              & 0.003                               & 0.438          & 0.136 & 2.075 & 123862 \\
    CG.B.8                 & 384.65                              & 0.004                               & 0.457          & 0.108 & 1.235 & 63777  \\
    MG.A.2                 & 112.27                              & 0.002                               & 0.846\tnote{a} & 0.237 & 3.930 & 236473 \\
    MG.A.4                 & 59.84                               & 0.003                               & 0.442          & 0.128 & 2.070 & 123875 \\
    MG.A.8                 & 31.38                               & 0.003                               & 0.476          & 0.114 & 1.041 & 60627  \\
    MG.B.2                 & 526.28                              & 0.002                               & 0.821          & 0.238 & 4.176 & 236635 \\
    MG.B.4                 & 280.11                              & 0.003                               & 0.432          & 0.130 & 1.706 & 123793 \\
    MG.B.8                 & 148.29                              & 0.003                               & 0.442          & 0.116 & 0.893 & 60600  \\
    LU.A.2                 & 2116.54                             & 0.002                               & 0.110          & 0.030 & 0.532 & 28754  \\
    LU.A.4                 & 1102.50                             & 0.002                               & 0.069          & 0.017 & 0.255 & 14915  \\
    LU.A.8                 & 574.47                              & 0.003                               & 0.067          & 0.016 & 0.192 & 8655   \\
    LU.B.2                 & 9712.87                             & 0.002                               & 0.357          & 0.104 & 1.734 & 101975 \\
    LU.B.4                 & 4757.80                             & 0.003                               & 0.190          & 0.056 & 0.808 & 53522  \\
    LU.B.8                 & 2444.05                             & 0.004                               & 0.222          & 0.057 & 0.548 & 30134  \\
    EP.A.2                 & 123.81                              & 0.002                               & 0.010          & 0.003 & 0.074 & 1834   \\
    EP.A.4                 & 61.92                               & 0.003                               & 0.011          & 0.004 & 0.073 & 1743   \\
    EP.A.8                 & 31.06                               & 0.004                               & 0.017          & 0.005 & 0.073 & 1661   \\
    EP.B.2                 & 495.49                              & 0.001                               & 0.009          & 0.003 & 0.196 & 2011   \\
    SP.B.4                 & 397.69                              & 0.002                               & 0.015          & 0.005 & 0.122 & 1763   \\
    SP.B.8                 & 196.74                              & 0.003                               & 0.018          & 0.006 & 0.082 & 1865   \\
    AA.A.2                 & 13.81                               & 0.002                               & 0.012          & 0.004 & 0.074 & 1362   \\
    AA.A.4                 & 6.92                                & 0.003                               & 0.011          & 0.003 & 0.073 & 1331   \\
    AA.A.8                 & 3.06                                & 0.004                               & 0.017          & 0.004 & 0.073 & 1225   \\
    AA.B.2                 & 49.49                               & 0.001                               & 0.009          & 0.003 & 0.196 & 2254   \\
    AA.B.4                 & 24.69                               & 0.002                               & 0.012          & 0.004 & 0.122 & 1453   \\
    AA.B.8                 & 12.74                               & 0.003                               & 0.018          & 0.005 & 0.082 & 1432   \\
    MG.A.2                 & 112.27                              & 0.002                               & 0.846          & 0.237 & 3.930 & 236473 \\
    MG.A.4                 & 59.84                               & 0.003                               & 0.442          & 0.128 & 2.070 & 123875 \\
    MG.A.8                 & 31.38                               & 0.003                               & 0.476          & 0.114 & 1.041 & 60627  \\
    MG.B.2                 & 526.28                              & 0.002                               & 0.821          & 0.238 & 4.176 & 236635 \\
    MG.B.4                 & 280.11                              & 0.003                               & 0.432          & 0.130 & 1.706 & 123793 \\
    MG.B.8                 & 148.29                              & 0.003                               & 0.442          & 0.116 & 0.893 & 60600  \\
    LU.A.2                 & 2116.54                             & 0.002                               & 0.110          & 0.030 & 0.532 & 28754  \\
    LU.A.4                 & 1102.50                             & 0.002                               & 0.069          & 0.017 & 0.255 & 14915  \\
    LU.A.8                 & 574.47                              & 0.003                               & 0.067          & 0.016 & 0.192 & 8655   \\
    LU.B.2                 & 9712.87                             & 0.002                               & 0.357          & 0.104 & 1.734 & 101975 \\
    LU.B.4                 & 4757.80                             & 0.003                               & 0.190          & 0.056 & 0.808 & 53522  \\
    LU.B.8                 & 2444.05                             & 0.004                               & 0.222          & 0.057 & 0.548 & 30134  \\
    EP.A.2                 & 123.81                              & 0.002                               & 0.010          & 0.003 & 0.074 & 1834   \\
    \bottomrule
  \end{longtable}
\end{ThreePartTable}



\subsection{算法环境}

算法环境可以使用 \pkg{algorithms} 宏包或者较新的 \pkg{algorithm2e} 实现。
\cref{algo:algorithm} 是一个使用 \pkg{algorithms} 的例子。关于排版算法环境
的具体方法,请阅读相关宏包的官方文档。

\begin{algorithm}
  \caption{计算斐波那契数列}\label{algo:algorithm}
  \begin{algorithmic}[1]
    \Require $n \geq 0$
    \Ensure $fib(n)$
    \Function{Fibonacci}{$n$}
    \If{$n \leq 1$}
    \State \Return $n$
    \Else
    \State \Return \Call{Fibonacci}{$n-1$} + \Call{Fibonacci}{$n-2$}
    \EndIf
    \EndFunction
  \end{algorithmic}
\end{algorithm}

\subsection{代码环境}

虽然我们可以在论文中加入算法,但是并不推荐添加过多的代码。如果确实需要在论文中添加代码,建议采用 \pkg{minted} 宏包,该宏包可以提供代码高亮的功能,使得代码更加易于阅读。

为了让代码可以在论文中正确地排版,我们使用了 \pkg{listings} 宏包来包装 \pkg{minted}。具体来说,我们在 \texttt{listing} 环境中嵌套了 \texttt{minted} 环境来展示代码,并使用 \texttt{caption} 命令来添加代码标题。代码的标签和交叉引用可以使用 \texttt{label} 和 \texttt{cref} 命令来实现。
\cref{algo:algorithm} 的Python代码如\cref{lst:fibonacci} 所示。

\begin{listing}[!htb]
  \begin{minted}{python}
def fibonacci(n: int) -> int:
    """
    Calculates the nth Fibonacci number.

    Args:
        n (int): Index of the desired Fibonacci number. Must be non-negative.

    Returns:
        int: The nth Fibonacci number.

    Raises:
        ValueError: If the input is negative.
    """
    if n < 0:
        raise ValueError("Negative arguments are not supported")
    if n == 0:
        return 0
    if n == 1:
        return 1
    return fibonacci(n - 1) + fibonacci(n - 2)
  \end{minted}
  \caption{计算斐波那契数列的Python实现}\label{lst:fibonacci}
\end{listing}

\subsection{\texttt{clearpage} 与 \texttt{newpage}}

在 \LaTeX\ 中,使用 \verb|\newpage| 命令可以在当前位置开始新的一页。而使用 \verb|\clearpage| 命令不仅可以开始新的一页,还可以将未处理的浮动体(例如图表等)全部输出,确保当前页面上的所有浮动体都已经处理完毕。
因此,在需要控制页面排版时,应该优先使用 \verb|\clearpage| 命令,而不是简单地使用 \verb|\newpage| 命令。如果不使用 \verb|\clearpage| 命令,可能会导致未处理的浮动体出现在不合适的位置,影响文档的排版效果。

另外,需要注意的是,如果在文档中使用了浮动体(例如图表等),那么最好将浮动体放在一个浮动体环境中,并设置适当的位置选项(例如 \verb|[htb]| 等),以控制浮动体的排版位置。在需要开始新的一页时,应该尽可能使用 \verb|\clearpage| 命令,以确保当前页面上的所有浮动体都已经处理完毕,从而避免出现排版问题。

综上所述,使用 \verb|\clearpage| 命令可以确保当前页面上的所有浮动体都已经处理完毕,从而避免出现排版问题。在文档中使用浮动体时,应该将浮动体放在一个浮动体环境中,并设置适当的位置选项,以控制浮动体的排版位置。