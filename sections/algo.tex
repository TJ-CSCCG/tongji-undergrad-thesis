\section{算法设计}\label{algorithm}
\subsection{ADMM算法}
\par ADMM最早在20世纪70年代被提出, 可见文\cite{Glowinski1975Sur,Gabay1976A}, 其源头可追溯到求解偏微分方程 (可见文\cite{Boyd2011Distributed}). 关于ADMM的全面详述及其应用可见文\cite{Boyd2011Distributed}.
\par ADMM算法求解问题的一般形式为
\begin{equation}
	\begin{array}{rl}
	\min\limits_{x\in\mathbb{R}^d} & \theta(x):=\sum\limits_{i=1}^n\theta_i(x_i)+\ell(x_1,\ldots,x_n)\\
	\st & \sum\limits_{i=1}^nA_ix_i=b,
	\end{array}
	\label{general problem for ADMM}
\end{equation}
这里$\theta_i:\mathbb{R}^{d_i}\mapsto(-\infty,+\infty](i=1,2,\ldots,n)$, $\ell:\mathbb{R}^d\mapsto(-\infty,+\infty]$; $x_i\in\mathbb{R}^{d_i}$; $A_i\in\mathbb{R}^{m\times d_i}$, $b\in\mathbb{R}^m$. 我们有问题\eqref{general problem for ADMM}的增广拉格朗日函数
$$\overline{\mcL}_A(x_1,\ldots,x_n,\bar\mu)=\sum\limits_{i=1}^n\theta_i(x_i)+\ell(x_1,\ldots,x_n)-\bar\mu^T\left(\sum\limits_{i=1}^nA_ix_i-b\right)+\frac{\bar\beta}{2}\left\Vert\sum\limits_{i=1}^nA_ix_i-b\right\Vert^2,$$
其中$\bar\mu\in\mathbb{R}^m$为拉格朗日乘子, $\bar\beta>0$为惩罚因子. 于是便有$n$块ADMM:
$$\begin{aligned}
	x_1^{k+1}&:=\arg\min\limits_{x_1\in\mathbb{R}^{d_1}}\overline{\mcL}_A(x_1,\ldots,x_n^k,\bar\mu^k),\\
	&\cdots\cdots\\
	x_n^{k+1}&:=\arg\min\limits_{x_n\in\mathbb{R}^{d_n}}\overline{\mcL}_A(x_1^{k+1},\ldots,x_n,\bar\mu^k),\\
	\bar\mu^{k+1}&:=\bar\mu^k-\bar\beta\left(\sum\limits_{i=1}^nA_ix_i^{k+1}-b\right).
\end{aligned}$$
\par 关于ADMM算法的收敛性分析大多聚焦于问题\eqref{general problem for ADMM}的特殊形式——两块凸可分问题, 即$n=2,\ell=0$, $\theta_1,\theta_2$都是凸函数. 此时, 在较弱的条件下, ADMM算法收敛; 可见文\cite{Boyd2011Distributed}. 在同样的条件下, 一些研究工作 (如文\cite{He2012On,Monteiro2013Iteration,Davis2014COnvergence})表明ADMM算法以$\mathcal{O}(\frac{1}{t})$或$o(\frac{1}{t})$的速度次线性收敛, 且在适当加速后, 速度达到$\mathcal{O}(\frac{1}{t^2})$ (见文\cite{Goldstein2014Fast,Goldfarb2012Fast}). 进一步地, 文\cite{Deng2012On}表示ADMM在目标函数和约束满足特定条件时是线性收敛的.
\par 对于$n\ge3$的情形, 大量的研究工作集中在将多块ADMM及其变体应用于带线性约束的可分凸优化问题上, 即问题\eqref{general problem for ADMM}去掉$\ell$, $\theta_i,i=1,\ldots,n$都是凸函数. 文\cite{Chen2016The}表明$n$块ADMM算法在某些病态问题上可能会发散. 因此, 多数研究的切入点, 或是在问题上添加额外的条件, 或是证明ADMM算法变体的收敛性. 一种典型的方式是合并$n$块ADMM的所有校正步 (见文\cite{He2012Alternating,He2012On,He2015A}). 另外, 若问题\eqref{general problem for ADMM}的目标函数中至少有$n-2$个函数是强凸的且惩罚因子被限制在一个特定的范围内, 则$n$块ADMM算法是全局收敛的 (见文\cite{Cai2017On,Chen2013On,Han2013An,Li2015A,Lin2015On,Zhang2010Convergence}). 如若没有强凸的条件, 则在目标函数满足特定的误差界条件时, 若以小步长更新对偶变量, 即乘子更新变为
$$\bar\mu^{k+1}=\bar\mu^k-\tau\bar\beta\left(\sum\limits_{i=1}^nA_ix_i^{k+1}-b\right),$$
则$n$块ADMM算法是线性收敛的 (见文\cite{Hong2017On}). 文\cite{Lin2016Iteration,Lin2015Onthe}则说明了多块ADMM算法在其他一些条件下的收敛性. 文\cite{Sun2015On}提出了多块ADMM算法的一种随机变体RPADMM. 在每一步, RPADMM构造$\{1,2\ldots,n\}$的一个随机排列, 然后按排列的顺序更新原始变量$x_i(i=1,2,\ldots,n)$. 令人惊喜的是, RPADMM对任意非奇异的方形线性系统都是按期望收敛的.
\par 相较于可分的情形, 在不可分的问题 (即使$n=2$)上的研究非常有限. 文\cite{Hong2016COnvergence}提到当目标函数对于变量来说是不可分时, 即使$n=2$且$\theta(\cdot)$为凸, 多块ADMM的收敛性仍然是开放的问题. 文\cite{Hong2014A}说明当问题\eqref{general problem for ADMM}是凸的但不可分且满足一定的误差界条件时, 若乘子的更新步长充分的小, 则ADMM迭代收敛到某个原始-对偶最优解. 尽管如此, 步长通常依赖于与误差界相关的一些未知参数, 因此很难计算, 这样反而不利于算法的高效实施. 因此, 往往直接使用经典的ADMM迭代 ($\tau=1$)或者其变体 ($\tau\ge1$). 文\cite{Cui2016On}讨论了求解有一耦合光滑目标函数的凸优化问题的一种优化ADMM算法. 文\cite{Gao2017First}研究了2块邻近ADMM及其变体在不可分凸优化问题上的收敛性和遍历复杂度, 其中假设了在问题数据上的一些附加条件. 文\cite{Chen2019Extended}则分析了求解不可分凸优化问题时, 邻近ADMM和RPADMM产生的序列的收敛性.
\par 对于目标是非凸的情形, ADMM算法的收敛性也依然不是很清晰. 尽管如此, ADMM算法在多种涉及非凸目标函数的应用上都具有极佳的表现, 例如非负矩阵分解 (可见文\cite{Zhang2010An,Sun2014Alternating})、分布式矩阵分解 (可见文\cite{Zhang2014Asynchronous})、分布式聚类 (可见文\cite{Forero2011Distributed})、张量分解 (可见文\cite{Liavas2013Parallel})、资产配置 (可见文\cite{Wen2013Asset})等. 然而现有的关于ADMM算法在非凸问题上的分析十分有限——所有已知的全局收敛性分析都需要在迭代序列上强加无法检验的条件. 例如, 文\cite{Jiang2013Alternating,Shen2014Augmented,Xu2011An,Wen2013Asset}表明在假设聚点存在以及相邻迭代点 (原始与对偶)的差趋近于0的条件下, ADMM在某些非凸问题上全局收敛到稳定点的集合. 但这样的假设条件限制过强. 同时, 去除在迭代点上的假设, ADMM是否还有相同的收敛结果还未可知. 文\cite{Zhang2010Convergence}分析了在特定非凸二次规划问题上的一族分裂算法 (其中ADMM为特殊情形), 并且说明它们在对偶步长满足一定条件时收敛到稳定解. 当然也有许多研究工作提出了求解非凸非光滑问题的新算法, 例如文\cite{Razaviyayn2013A,Ghadimi2015Accelerated,Ghadimi2014Mini,Scutari2014Decomposition,Bolte2014Proximal}. 但它们都不能处理带线性耦合约束的非凸问题. 文\cite{Hong2016COnvergence}建立了ADMM对某些特定类型的非凸问题的收敛性, 其中没对序列作任何假设. 其结果表明, 只要目标函数中的$\theta_i$'s和$\ell$满足特定的正则条件, 且惩罚因子$\bar\beta$选取得足够大, 则由ADMM算法产生的迭代点列收敛到问题的稳定点. 
\par 以上工作中, 与问题\eqref{original problem matrix form 2}最契合的是文\cite{Hong2016COnvergence}中考虑的非凸一致性问题:
$$\begin{array}{rl}
	\min\limits_x & \sum\limits_{i=1}^n\theta_i(x)+\ell(x),\\
	\st & x\in \mathcal{X},
\end{array}$$
其中$\theta_i$'s为光滑(可能)非凸的函数, $\ell$为凸的非光滑正则化项. 在我们的问题\eqref{original problem matrix form 2}中, 显然$\theta(X)=2\langle X,R\rangle$, $\ell(X)=\langle X,XR\rangle$. $\ell(X)$非凸. 因此文\cite{Hong2016COnvergence}中的结论不可直接使用.
\par 基于ADMM算法在许多非凸问题上的良好效果, 我们使用ADMM算法求解问题\eqref{original problem matrix form}. 注意第\ref{preliminaries}节中引入分裂变量方便了ADMM算法的设计. 此外, 将原本问题\eqref{original problem matrix form 2}目标函数中的二次项变为双线性项有利于ADMM子问题的求解. 这点我们会在后文说明. 问题\eqref{original problem matrix form}的增广拉格朗日函数为
\begin{equation}
	\mcL_A(X,Z,\Phi)=f(X,Z)-\langle\Phi,X-Z\rangle+\frac{\beta}{2}\Vert X-Z\Vert_F^2.
	\label{alf}
\end{equation}
这里$\beta\ge0$为惩罚因子. 下面我们有对于问题\eqref{original problem matrix form}ADMM算法的一般框架. 见框架\ref{ADMM framework}.
\begin{algorithm}[htbp]
\textbf{输入: }$X^0,Z^0,\Phi^0,\beta^0$\\
\textbf{输出: }$X^k,Z^k,\Phi^k$
\floatname{algorithm}{框架}
\caption{求解问题\eqref{original problem matrix form}的ADMM算法框架}
\label{ADMM framework}
\begin{algorithmic}[1]
	\WHILE{收敛性测试未满足}
	\STATE{$X^{k+1}=\arg\min\limits_{X:X\one=\rho,\trace(X)=0}\mcL_A(X,Z^k,\Phi^k);$}
	\STATE{$Z^{k+1}=\arg\min\limits_{Z:Z^T\one=\rho,Z\ge0}\mcL_A(X^{k+1},Z,\Phi^k);$}
	\STATE{更新$\Phi^k$得到$\Phi^{k+1}$;}
	\STATE{如有需要, 更新$\beta^k$得到$\beta^{k+1}$;}
	\STATE{$k:=k+1$;}
	\ENDWHILE
\end{algorithmic}

\end{algorithm}
我们在本节剩下的部分将详细介绍如何求解框架\ref{ADMM framework}中$X,Z$子问题的求解以及乘子 (或对偶变量)的更新方式.

\subsection{$X$子问题}
设当前我们在第$k$步迭代. 下面我们重述$X$子问题. 我们省略上标$k$, 并用``$+$''表示新迭代点.
\begin{equation}
	\begin{array}{rl}
		\min\limits_X & 2\langle X,R\rangle+\langle Z,XR\rangle-\langle\Phi,X-Z\rangle+\frac{\beta}{2}\Vert X-Z\Vert_F^2\\
		\st & X\one=\rho,\quad\trace(X)=0.
	\end{array}
	\label{X subproblem}
\end{equation}
上述问题实质上是一个带等式约束的凸二次规划. 因此我们可以直接使用拉格朗日乘数法得到解析解. 问题\eqref{X subproblem}的拉格朗日函数为
\begin{equation}
	\begin{aligned}
		\mcL_0^X=&2\langle X,R\rangle+\langle Z,XR\rangle-\langle\lambda_1,X\one-\rho\rangle-\mu\trace(X)\\
		&-\langle\Phi,X-Z\rangle+\frac{\beta}{2}\Vert X-Z\Vert_F^2,
	\end{aligned}
	\label{lagrangian X subproblem}
\end{equation}
其中$\mcL$的上标$X$表示其对$X$子问题的依赖 (下对$Z$子问题同), $\mu\in\mathbb{R},\lambda_1\in\mathbb{R}^n$为乘子. 由表达式\eqref{lagrangian X subproblem}, 直接计算可得一阶充要最优性条件:
\begin{subequations}
	\begin{numcases}{}
		\nabla_X\mcL_0^X=\beta X+(2R+ZR-\lambda_1\one^T-\mu I-\Phi-\beta Z)=0,\label{fonc X 1}\\
		\nabla_{\mu}\mcL_0^X=-\trace(X)=0,\label{fonc X 2}\\
		\nabla_{\lambda_1}\mcL_0^X=-(X\one-\rho)=0.\label{fonc X 3}
	\end{numcases}
\end{subequations}
由等式\eqref{fonc X 1}, 我们有
\begin{equation}
	X^+=-\frac{1}{\beta}(2R+ZR-\lambda_1\one^T-\mu I-\Phi-\beta Z).
	\label{pre solve X}
\end{equation}
结合公式\eqref{pre solve X}和等式\eqref{fonc X 2},\eqref{fonc X 3}, 我们有
$$\begin{aligned}
	0&=\trace(X)\\
	&=-\frac{1}{\beta}(2\trace(R)+\trace(ZR)-\trace(\Phi)-\beta\trace(Z)-\one^T\lambda_1-n\mu)\\
	&=-\frac{1}{\beta}(2\trace(R)+\trace(ZR)-\trace(\Phi)-\beta\trace(Z))+\frac{1}{\beta}(\one^T\lambda_1+n\mu),\\
	\rho&=X\one\\
	&=-\frac{1}{\beta}(2R+ZR-\Phi-\beta Z-\lambda_1\one^T-\mu I)\one\\
	&=-\frac{1}{\beta}(2R\one+ZR\one-\Phi \one-\beta Z\one)+\frac{1}{\beta}(n\lambda_1+\one\mu),
\end{aligned}$$
由此得到线性方程组
\begin{equation}
	\begin{pmatrix}
		nI & \one\\\one^T n
	\end{pmatrix}\begin{pmatrix}
		\lambda_1\\\mu
	\end{pmatrix}=\begin{pmatrix}
		M_1\\m_1
	\end{pmatrix},
	\label{ls}
\end{equation}
其中
$$\begin{aligned}
	M_1&=2R\one+ZR\one-\Phi \one-\beta Z\one+\beta\rho,\\
	m_1&=2\trace(R)+\trace(ZR)-\trace(\Phi)-\beta\trace(Z).
\end{aligned}$$
线性方程组\eqref{ls}有解析解:
\begin{equation}
	\mu=\frac{1}{n-1}\left(-\frac{1}{n}\one^TM_1+m_1\right),\quad\lambda_1=\frac{1}{n}(M_1-\one\mu).
	\label{multiplier X subproblem}
\end{equation}
最终由公式\eqref{pre solve X}和公式\eqref{multiplier X subproblem}给出问题\eqref{X subproblem}解的具体形式. 

\subsection{$Z$子问题}\label{Z}
我们重述$Z$子问题如下.
\begin{equation}
	\begin{array}{rl}
		\min\limits_Z & \langle Z,X^+R\rangle-\langle\Phi,X^+-Z\rangle+\frac{\beta}{2}\Vert X^+-Z\Vert_F^2\\
		\st & Z^T\one=\rho,\quad Z\ge0.
	\end{array}
	\label{Z subproblem}
\end{equation}
类似地, 问题\eqref{Z subproblem}也是个凸二次规划. 求解之的主要困难在于不等式约束. 忽略非负约束, 问题\eqref{Z subproblem}实际上就是要求我们在超平面$Z^T\one=\rho$上沿着目标函数的下降方向寻找可行解. 且在即将打破非负约束的地方停止搜索, 即得问题\eqref{Z subproblem}的解. 求解凸二次规划可直接调用MATLAB的内置函数\texttt{quadprog()}. 考虑到问题\eqref{Z subproblem}形式略为复杂, 之后我们将考虑问题\eqref{Z subproblem}的等价形式问题\eqref{equal Z 1}. 其间的讨论和推导可作为它们等价性的说明. 我们将最后说明求解问题\eqref{equal Z 1}等价于求解有限个互不相关的简单二次规划. 
\par 问题\eqref{Z subproblem}$\Leftrightarrow$问题\eqref{equal Z 1}. 在超平面$Z^T\one=\rho$上寻求最优解, 我们可先在超平面上找到一点. 我们暂且忽略非负约束, 而仅考虑问题\eqref{Z subproblem}中的等式约束. 于是我们有问题\eqref{Z subproblem}的拉格朗日函数
\begin{equation}
	\mcL_0^Z=\langle Z,X^+R\rangle-\langle\Phi,X^+-Z\rangle+\frac{\beta}{2}\Vert X^+-Z\Vert_F^2-\langle\lambda_2,Z^T\one-\rho\rangle,
	\label{lagrangian Z subproblem}
\end{equation}
其中$\lambda_2\in\mathbb{R}^n$为乘子. 由表达式\eqref{lagrangian Z subproblem}, 直接计算可得一阶充要最优性条件:
\begin{subequations}
	\begin{numcases}{}
		\nabla_Z\mcL_0^Z=\beta Z+(X^+R+\Phi-\beta X^+-\one\lambda_2^T)=0,\label{fonc Z 1}\\
		\nabla_{\lambda_2}\mcL_0^Z=-(Z^T\one-\rho)=0.\label{fonc Z 2}
	\end{numcases}
\end{subequations}
由等式\eqref{fonc Z 1}我们有
\begin{equation}
	Z=-\frac{1}{\beta}(X^+R+\Phi-\beta X^+-\one\lambda_2^T).
	\label{pre solve Z}
\end{equation}
结合公式\eqref{pre solve Z}和等式\eqref{fonc Z 2}, 我们有
$$\rho=Z^T\one=-\frac{1}{\beta}\left(R\left(X^+\right)^T\one+\Phi^T\one-\beta\left(X^+\right)^T\one-n\lambda_2\right),$$
这就给出
\begin{equation}
	\lambda_2=\frac{1}{n}\left[R\left(X^+\right)^T\one+\Phi^T\one-\beta\left(X^+\right)^T\one+\beta\rho\right].
	\label{multiplier Z subproblem}
\end{equation}
由公式\eqref{pre solve Z}和公式\eqref{multiplier Z subproblem}我们得到了等式约束下的解. 我们记之为$\widetilde Z$. 于是我们得到问题\eqref{Z subproblem}的等价形式:
\begin{equation}
	\begin{array}{rl}
		\min\limits_Z & \Vert Z-\widetilde Z\Vert_F^2\\
		\st & Z^T\one=\rho,\quad Z\ge0.
	\end{array}
	\label{equal Z 1}
\end{equation}
\par 以上问题\eqref{equal Z 1}是方便用现有算法求解的. 事实上, 对$Z,\widetilde Z$做列分块, 
$$Z=[z_1,\ldots,z_n],\quad\widetilde Z=[\tilde z_1,\ldots,\tilde z_n].$$
那么问题\eqref{equal Z 1}就可以写成
\begin{equation}
	\begin{array}{rl}
	\min\limits_{z_1,\ldots,z_n} & \sum\limits_{j=1}^n\left\Vert z_j-\tilde z_j\right\Vert^2\\
	\st & \one^Tz_j=\rho_j,\quad z_j\ge0,\quad j=1,\ldots,n.
	\end{array}
	\label{column}
\end{equation}
显然, 问题\eqref{column}可以进一步分拆成$n$个互补相关的列子问题. 对每个列子问题, 由于已经是向量形式, 我们就可以直接用MATLAB内置函数\texttt{quadprog()}求解. 记更新后的$Z$为$Z^+$.

\begin{rem}
	以上两子问题的求解在没有将问题\eqref{original problem matrix form 2}目标函数的二次项替换成双线性项时是无法想象的.
\end{rem}

\subsection{拉格朗日乘子更新}\label{update multiplier1}
对于一般的等式约束非线性优化问题
\begin{equation}\begin{array}{rl}
\min\limits_x & \bar f(x)\\
\st & c_i(x)=0,\quad i\in\mathcal{E},
\end{array}\label{nlp}\end{equation}
其中$x\in\mathbb{R}^n$, $\mathcal{E}$表示等式约束的指标集. 于是有增广拉格朗日函数
$$\widetilde\mcL_A(x,\lambda)=\bar f(x)-\sum\limits_{i\in\mathcal{E}}\lambda_ic_i(x)+\frac{\gamma}{2}\sum\limits_{i\in\mathcal{E}}c_i^2(x).$$
这里$\lambda_i\in\mathbb{R}^n$为拉格朗日乘子, $\gamma$为惩罚因子. 令
$$x^k\in\arg\min\limits_x\widetilde\mcL_A(x,\lambda^k),$$
我们有此无约束极小问题的最优性条件:
$$0=\nabla_x\widetilde\mcL_A(x^k,\lambda^k)=\nabla \bar f(x^k)-\sum\limits_{i\in\mathcal{E}}\left[\lambda_i^k-\gamma^kc_i(x^k)\right]\nabla c_i(x^k).$$
比较上式与问题\eqref{nlp}的KKT条件, 我们推出
\begin{equation}\lambda_i^*\approx\lambda_i^k-\gamma^kc_i(x^k),\quad\forall i\in\mathcal{E},\label{update multiplier}\end{equation}
这里$\lambda^*$表示最优拉格朗日乘子. 文\cite{Nocedal2006Numerical}表示, 公式\eqref{update multiplier}可作为增广拉格朗日函数法的乘子更新策略. 于是, 我们的ADMM算法中的乘子更新步可选为
\begin{equation}
\Phi^+=\Phi-\beta(X^+-Z^+).
\label{update ADMM multiplier X block}
\end{equation}
进一步, 我们可以加入松弛因子$\alpha>0$:
\begin{equation}
\Phi^+=\Phi-\alpha\beta(X^+-Z^+).
\label{relax update X block}
\end{equation}

\subsection{停机准则与KKT违反度}
我们回顾问题\eqref{original problem matrix form}的KKT条件\eqref{KKT}: 若$(X^*,Z^*)$为问题\eqref{original problem matrix form}的解, 则存在拉格朗日乘子$\mu^*,\lambda_1^*,\lambda_2^*,\Phi^*,0\le\Omega^*$, 使得
\begin{subequations}
	\begin{numcases}{}
	2R+Z^*R-\lambda_1^*\one^T-\mu^*I-\Phi^*=0,\label{X block kkt1}\\
	X^*R-\one\left(\lambda_2^*\right)^T+\Phi^*-\Omega^*=0,\label{X block kkt2}\\
	X^*\one=\rho,\trace(X^*)=0,\label{X block kkt3}\\
	\left(Z^*\right)^T\one=\rho,Z^*\ge0,\label{X block kkt4}\\
	\Omega^*\ge0,\label{X block kkt5}\\
	X^*=Z^*,\label{X block kkt6}\\
	\Omega^*\circ Z^*=0.\label{X block kkt7}
	\end{numcases}
	\label{KKT}
\end{subequations}
注意在第$k$次迭代, $X^{k+1}$是问题\eqref{X subproblem}的解, 因此存在拉格朗日乘子$\lambda_1^{k+1},\mu^{k+1}$使得
\begin{subequations}
	\begin{numcases}{}
	2R+Z^kR-\Phi^k+\beta(X^{k+1}-Z^k)-\lambda_1^{k+1}\one^T-\mu^{k+1}I=0,\label{X block X subproblem kkt1}\\
	X^{k+1}\one=\rho,\trace(X^{k+1})=0,\label{X block X subproblem kkt2}
	\end{numcases}
\end{subequations}
其中等式\eqref{X block X subproblem kkt1},\eqref{X block X subproblem kkt2}分别对应于等式\eqref{X block kkt1},\eqref{X block kkt3}. 特别地, 若使用更新策略\eqref{update ADMM multiplier X block}, 
$$\begin{aligned}
2R&+Z^kR-\Phi^k+\beta(X^{k+1}-Z^k)-\lambda_1^{k+1}\one^T-\mu^{k+1}I\\
&=2R+Z^{k+1}R-\Phi^{k+1}-\lambda_1^{k+1}\one^T-\mu^{k+1}I+(Z^{k+1}-Z^k)(\beta I-R),
\end{aligned}$$
这说明我们必须多加关注$(Z^{k+1}-Z^k)(\beta I-R)$. 若使用更新策略\eqref{relax update X block}, 则
$$\begin{aligned}
2R&+Z^kR-\Phi^k+\beta(X^{k+1}-Z^k)-\lambda_1^{k+1}\one^T-\mu^{k+1}I\\
&=R+Z^{k+1}R-\Phi^{k+1}-\lambda_1^{k+1}\one^T-\mu^{k+1}I+(Z^{k+1}-Z^k)(\beta I-R)+(1-\alpha)\beta(X^{k+1}-Z^{k+1}),
\end{aligned}$$
这说明我们还需要多考虑$(1-\alpha)\beta(X^{k+1}-Z^{k+1})$. 
而对于$Z$子问题, $Z^{k+1}$是问题\eqref{Z subproblem}的解, 于是存在拉格朗日乘子$\lambda_2^{k+1},\Omega^{k+1}\ge0$使得
\begin{subequations}
	\begin{numcases}{}
	X^{k+1}R-\Phi^k-\beta(X^{k+1}-Z^{k+1})-\one\left(\lambda_2^{k+1}\right)^T-\Omega^{k+1}=0,\label{X block U subproblem kkt1}\\
	\left(Z^{k+1}\right)^T\one=\rho,Z^{k+1}\ge0,\label{X block U subproblem kkt2}\\
	\Omega^{k+1}\ge0,\label{X block U subproblem kkt3}\\
	\Omega^{k+1}\circ Z^{k+1}=0,\label{X block U subproblem kkt4}
	\end{numcases}
\end{subequations}
其中等式\eqref{X block U subproblem kkt1}-\eqref{X block U subproblem kkt4}分别对应于等式\eqref{X block kkt2},\eqref{X block kkt4},\eqref{X block kkt5},\eqref{X block kkt7}. 若使用更新策略\eqref{update ADMM multiplier X block}, 则易知$\left(Z^{k+1},\lambda_2^{k+1},\Omega^{k+1},\Phi^{k+1}\right)$自动满足KKT条件. 事实上, 
$$\begin{aligned}
X^{k+1}R-\Phi^k&-\beta(X^{k+1}-Z^{k+1})-\one\left(\lambda_2^{k+1}\right)^T-\Omega^{k+1}\\
&=X^{k+1}R-\Phi^{k+1}-\one\left(\lambda_2^{k+1}\right)^T-\Omega^{k+1}.
\end{aligned}$$
若使用更新策略\eqref{relax update X block}, 则有
$$\begin{aligned}
X^{k+1}R-\Phi^k&-\beta(X^{k+1}-Z^{k+1})-\one\left(\lambda_2^{k+1}\right)^T-\Omega^{k+1}\\
&=X^{k+1}R-\Phi^{k+1}-\one\left(\lambda_2^{k+1}\right)^T-\Omega^{k+1}-(1-\alpha)\beta(X^{k+1}-Z^{k+1}).
\end{aligned}$$
于是再次地, 我们需考虑$(1-\alpha)\beta(X^{k+1}-Z^{k+1})$. 
\par 总之, KKT违反度可用
\begin{equation}
t^{k+1}\triangleq\Vert X^{k+1}-Z^{k+1}\Vert_{\infty},\quad s^{k+1}\triangleq\Vert (Z^{k+1}-Z^k)(\beta I-R)\Vert_{\infty}.
\label{X block residual}
\end{equation}
度量. 我们分别称$t^{k+1},s^{k+1}$为原始残差和对偶残差. 因此, 我们可在以下情形终止内循环:
\begin{enumerate}[\textbf{情形} 1]
	\item $t^{k+1},s^{k+1}$都足够小; 或
	\item 对某个$p^{k+1}\in(0,1)$, $p^{k+1}s^{k+1}+\left(1-p^{k+1}\right)t^{k+1}$足够小. 其中$p^{k+1}$反映了我们主观上对两类残差所赋的权值. 默认情形下, $p^{k+1}=0.5$.
\end{enumerate}

\subsection{完整算法}
我们使用
\begin{equation}E^{k+1}=(1-p^{k+1})t^{k+1}+p^{k+1}s^{k+1},\label{KKT violation of inner loop X block}\end{equation}
作为KKT违反度. 
\begin{breakablealgorithm}
	\floatname{breakablealgorithm}{算法}
	\caption{求解问题\eqref{original problem matrix form}的ADMM算法}
	\label{BCD-ADMM}
	\begin{algorithmic}[1]
		\REQUIRE 初始值$X^0,Z^0,\Phi^0$, 初始惩罚因子$\beta^0$, 容忍限$\epsilon$, 初始化迭代数$k=0$, 初始化对偶残差和原始残差$s^0:=1,t^0:=1$, 松弛因子$\alpha>0 (\makebox{默认值为}1)$, 初始化权重$p^0\in(0,1)$.
		\ENSURE $X^k,Z^k,\Phi^k$
		\STATE $E^k=(1-p^{k})t^k+p^ks^k$;
		\WHILE{$E^k>\epsilon$}
		\STATE\texttt{*求解$X$子问题}
		\STATE {由公式\eqref{multiplier X subproblem}计算$\mu^{k+1},\lambda_1^{k+1}$;}
		\STATE {由公式\eqref{pre solve X}计算$X^{k+1}$;}
		\STATE\texttt{*求解$Z$子问题}
		\STATE {使用MATLAB内置函数\texttt{quadprog()}求解列子问题\eqref{column}得到$Z^{k+1}$;}
		\STATE\texttt{*更新乘子}
		\STATE {$\Phi^{k+1}=\Phi^k-\alpha\beta^k(X^{k+1}-Z^{k+1})$;}
		\STATE\texttt{*更新残差与各因子}
		\STATE{由公式\eqref{X block residual}计算原始和对偶残差$t^{k+1},s^{k+1}$;}
		\STATE{更新$\beta^k$得到$\beta^{k+1}$;}
		\STATE{更新$p^k$得到$p^{k+1}$;}
		\STATE{由公式\eqref{KKT violation of inner loop X block}更新KKT违反度得到$E^{k+1}$;}
		\STATE{$k:= k+1$;}
	\ENDWHILE
	\end{algorithmic}
\end{breakablealgorithm}

\newpage
\section{收敛性分析}
\label{convergence analysis}
下面的定理\ref{convergence of ADMM}表明, 在迭代序列收敛且满足一定条件时, 算法\ref{BCD-ADMM}收敛到问题\eqref{original problem matrix form}的稳定点. 
\begin{thm}\label{convergence of ADMM}
	假设算法\ref{BCD-ADMM}每一步$X,Z$问题均精确求解, 且产生的迭代序列$\{X^k\},\{Z^k\},\{\Phi^k\}$分别收敛到$X^*,Z^*,\Phi^*$, 满足$X^*=Z^*$. 则$(X^*,Z^*,\Phi^*)$为问题\eqref{original problem matrix form}的稳定点.
\end{thm}
\begin{pf}
	回顾KKT条件\eqref{KKT}, 我们只需要证明如下2条, 便能由它们的KKT条件和$X^*=Z^*$证明定理. 
	\begin{enumerate}
	\item $X^*\in\arg\min\limits_{X\one=\rho,\trace(X)=0}f(X,Z^*)-\langle\Phi^*,X-Z^*\rangle.$\label{convergence1}
	\item $Z^*\in\arg\min\limits_{Z^T\one=\rho,Z\ge0}f(X^*,Z)-\langle\Phi^*,X^*-Z\rangle.$\label{convergence2}
	\end{enumerate}
	由于$X$子问题精确求解, 所以
	$$X^{k+1}=\arg\min\limits_{X\one=\rho,\trace(X)=0}f(X,Z^k)-\langle\Phi^k,X-Z^k\rangle+\frac{\beta}{2}\Vert X-Z^k\Vert_F^2,$$
	且
	$$\begin{aligned}
		&f(X,Z^k)-\langle\Phi^k,X-Z^k\rangle+\frac{\beta}{2}\Vert X-Z^k\Vert_F^2\\
		&=\frac{\beta}{2}\Vert X\Vert_F^2-\langle X,\beta Z^k-Z^kR-2R+\Phi^k\rangle+\frac{\beta}{2}\Vert Z^k\Vert_F+\langle\Phi^k,Z^k\rangle\\
		&=\frac{\beta}{2}\left\Vert X-\frac{1}{\beta}(\beta Z^k-Z^kR-2R+\Phi^k)\right\Vert_F^2+C,
	\end{aligned}$$
	其中$C$相对于$X$是常数. 考虑到集合$\{X:X\one=\rho,\trace(X)=0\}$的凸性, 根据文\cite{Rockafellar2015Convex}, 我们得出
	\begin{equation}\left\langle\frac{1}{\beta}(\beta Z^k-Z^kR-2R+\Phi^k)-X^{k+1},X-X^{k+1}\right\rangle\le0\label{variation}\end{equation}
	对任何$X:X\one=\rho,\trace(X)=0$都是成立的. 由于
	$$\begin{aligned}f(X,Z^k)-f(X^{k+1},Z^k)&=\left(2\langle X,R\rangle+\langle Z^k,XR\rangle\right)-\left(2\langle X^{k+1},R\rangle+\langle Z^k,X^{k+1}R\rangle\right)\\
	&=\langle 2R+Z^kR,X-X^{k+1}\rangle,\end{aligned}$$
	所以不等式\eqref{variation}等价于
	$$f(X,Z^k)-f(X^{k+1},Z^k)-\left\langle\Phi^k-\beta(X^{k+1}-Z^k),X-X^{k+1}\right\rangle\ge0.$$
	两边取极限$k\to\infty$, 我们有
	$$f(X,Z^*)-f(X^*,Z^*)-\langle\Phi^*,X-X^*\rangle\ge0.$$
	利用$X^*=Z^*$, 我们就完成了第\ref{convergence1}条的证明:
	$$f(X,Z^*)-\langle\Phi^*,X-Z^*\rangle\ge f(X^*,Z^*)-\langle\Phi^*,X^*-Z^*\rangle.$$
	第\ref{convergence2}条的证明过程类似. 
\end{pf}