\section{引用}\label{sec:reference}

在本节(\cref{sec:reference})中,我们将探讨如何在 \LaTeX\ 中进行参考文献引用和交叉引用,以便于读者查阅文献和方便地引用文档中的其他部分。

\subsection{参考文献的引用}

在学术论文或科技报告中,通常需要引用相关文献以支持观点或论证。为了方便读者查阅,我们需要在论文中标注参考文献。在 \LaTeX\ 中,可使用 \verb|\cite| 命令引用参考文献。参考文献的外观应符合国标 GB/T 7714,本模板提供两种方式:
\begin{enumerate}
    \item 使用 \BibTeX{} 配合 \pkg{gbt7714} 宏包
    \item 使用 \BibLaTeX{} 配合 \pkg{biblatex-gb7714-2015} 样式包
\end{enumerate}
两种方式都能实现符合国标的参考文献格式。

在正文中,可使用 \verb|\cite| 命令引用参考文献。例如:

\begin{Verbatim}
这是一个引用示例 \cite{book1}。
\end{Verbatim}

其中,\verb|book1| 是参考文献的键值,即参考文献文件中每个条目的唯一标识符。引用命令会自动在文中插入相应文献序号,并在文末的参考文献列表中显示相应文献条目。

如果需要同时标注多个参考文献,可使用逗号分隔键值。例如:

\begin{Verbatim}
这是一个引用示例 \cite{book1,online1}。
\end{Verbatim}

请注意,本模板所用的所有文献均为ChatGPT生成的。我们不保证这些文献的真实性。

使用 \verb|\cite{key1,key2,key3...}| 命令可在正文中产生带有括号的上标引用的参考文献,如\cite{book1,online1,article1}。以下是使用 \verb|\cite| 命令的引用示例:
\begin{itemize}
  \item 普通图书\cite{book1,book2}
  \item 论文集、会议录\cite{conf1,conf2}
  \item 科技报告\cite{techreport1,techreport2}
  \item 学位论文\cite{thesis1,thesis2,thesis3}
  \item 专利文献\cite{patent1,patent2}
  \item 专著中析出的文献\cite{inbook1,inbook2}
  \item 期刊中析出的文献\cite{qin2021,article1,article2}
  \item 报纸中析出的文献\cite{newspaper1,newspaper2}
  \item 电子文献\cite{online1,online2,online3}
\end{itemize}

% 根据使用的是 BibTeX 还是 BibLaTeX,parentcite 命令不同
\ifx\parencite\undefined
% 如果使用BibTeX,使用citep
使用 \verb|\citep{key1,key2,key3...}| 命令可以在正文中产生带有括号的引用参考文献。下面是使用 \verb|\citep| 命令的引用示例:
\begin{itemize}
  \item 普通图书\citep{book1,book2}
  \item 论文集、会议录\citep{conf1,conf2}
  \item 科技报告\citep{techreport1,techreport2}
  \item 学位论文\citep{thesis1,thesis2,thesis3}
  \item 专利文献\citep{patent1,patent2}
  \item 专著中析出的文献\citep{inbook1,inbook2}
  \item 期刊中析出的文献\citep{qin2021,article1,article2}
  \item 报纸中析出的文献\citep{newspaper1,newspaper2}
  \item 电子文献\citep{online1,online2,online3}
\end{itemize}
\else
% 如果使用BibLaTeX,使用parencite
使用 \verb|\parencite{key1,key2,key3...}| 命令可以在正文中产生带有括号的引用参考文献。下面是使用 \verb|\parencite| 命令的引用示例:
\begin{itemize}
  \item 普通图书\parencite{book1,book2}
  \item 论文集、会议录\parencite{conf1,conf2}
  \item 科技报告\parencite{techreport1,techreport2}
  \item 学位论文\parencite{thesis1,thesis2,thesis3}
  \item 专利文献\parencite{patent1,patent2}
  \item 专著中析出的文献\parencite{inbook1,inbook2}
  \item 期刊中析出的文献\parencite{qin2021,article1,article2}
  \item 报纸中析出的文献\parencite{newspaper1,newspaper2}
  \item 电子文献\parencite{online1,online2,online3}
\end{itemize}
\fi

可使用 \verb|\nocite{key1,key2,key3...}| 将参考文献条目加入文献表中,但不在正文中引用。使用 \verb|\nocite{*}| 可将参考文献数据库中的所有条目加入文献表中。

% 以下部分仅在使用BibLaTeX时显示
\ifx\fullcite\undefined
% 使用BibTeX时,不显示这部分内容
\else
% 使用BibLaTeX时,显示这部分内容
\subsubsection{\BibLaTeX{}特有的引用命令}

当使用BibLaTeX宏包时,有一些额外的引用命令可供使用。

\paragraph{完整引用}

当我们想在正文(非参考文献章节)中插入对某一参考文献的完整引用时,可以使用 \verb|\fullcite{key1}| 命令。下面是使用 \verb|\fullcite| 命令的引用示例:

\begin{itemize}
    \item \fullcite{qin2021}
\end{itemize}

\paragraph{脚注引用}

有时,我们想要用脚注的形式引用某参考文献,那么我们可以使用 \verb|\footfullcite{key1}| 命令\footfullcite{qin2021}。
\fi

另外,为了避免遗漏引用,有时候需要使用文献管理工具来管理参考文献,例如 Zotero 或 EndNote 等。这些工具可以帮助你轻松管理文献数据库,生成参考文献列表,甚至将参考文献直接插入到文档中。
无论使用何种方法,都应该注意文献的准确性和完整性。在引用文献时,应尽可能使用最新的版本,并注意对文献信息的正确引用和格式化。这样可以提高文献的可信度,也能为读者提供更多参考信息。

\subsection{脚注}

脚注是一种在文本底部添加注释或补充说明的方式。在 \LaTeX\ 中,可以使用 \verb|\footnote| 命令添加脚注。例如,在文本中需要添加脚注时,可以在需要添加脚注的单词或句子后面使用 \verb|\footnote| 命令,如下所示:

\begin{Verbatim}
脚注是一种在文本底部添加注释或补充说明的方式\footnote{通常,我们在脚注里也写完整的句子。在文本中使用脚注时,应该遵循学术规范,尽可能引用可信的来源,并注明出处。}。
\end{Verbatim}

其中,花括号中的文本就是脚注的内容。编译文档后,脚注会出现在页面底部,并自动标上数字:脚注是一种在文本底部添加注释或补充说明的方式\footnote{通常,我们在脚注里也写完整的句子。在文本中使用脚注时,应该遵循学术规范,尽可能引用可信的来源,并注明出处。}。

需要注意的是,在使用脚注时,应该尽量避免使用过多的脚注,以免影响文本的阅读体验。同时,脚注的内容应该尽可能简洁明了,突出重点,有助于读者理解和记忆文本内容。
另外,在文本中使用脚注时,应该遵循学术规范,尽可能引用可信的来源,并注明出处。同时,对于自己的观点和推断,应该明确标注为个人观点,以免误导读者。

\subsection{交叉引用}

在文档中,交叉引用是指引用文档中的某个标签或标记,例如章节、图表、公式或页码等。在 \LaTeX\ 中,可以使用 \verb|\label| 命令为文档中的对象添加标签,使用 \verb|\ref| 命令或 \verb|\pageref| 命令进行引用。

例如,在文档中的某个章节中添加一个标签:

\begin{Verbatim}
\section{引言}
\label{sec:intro}
这是一段引言。
\end{Verbatim}

然后,在文档的其他位置使用 \verb|\ref| 命令来引用这个标签:

\begin{Verbatim}
请参见第 \ref{sec:intro} 章节。
\end{Verbatim}

这样,就可以在文档中产生“请参见第 X 章节”的效果,其中 X 为标签所在章节的编号。

类似地,使用 \verb|\pageref| 命令可以引用页面编号。例如:

\begin{Verbatim}
请参见第 \pageref{sec:intro} 页。
\end{Verbatim}

这样,就可以在文档中产生“请参见第 X 页”的效果,其中 X 为标签所在页码的编号。

交叉引用在文档中很常用,可以帮助读者快速定位到相关内容,提高文档的可读性。但是,需要注意标签的唯一性和正确性,以及引用命令的正确使用方式。在使用交叉引用时,建议先编译一次文档,然后再编译一次,以确保标签和引用命令的对应关系正确。

当文档中需要交叉引用多个对象时,可以使用 \verb|\cref| 命令来自动根据引用对象的类型生成正确的引用词汇。本模板已经导入了 \pkg{cleveref} 宏包,设定好了一组默认的引用词汇,这些引用词汇可以根据需要进行修改。

在文档中为需要引用的对象添加标签,例如:

\begin{Verbatim}
\section{引言}
\label{sec:intro}
这是一段引言。
\end{Verbatim}

接下来,在文档的其他位置使用 \verb|\cref| 命令引用这个标签:
\begin{Verbatim}
请参见\cref{sec:intro}。
\end{Verbatim}
这样,就可以在文档中产生“请参见章节 X”的效果,其中 X 为标签所在章节的编号。\pkg{cleveref} 宏包会自动根据标签所在对象的类型生成正确的引用词汇。

需要注意的是,如果需要修改引用词汇,可以在导言区添加如下代码:

\begin{Verbatim}
\crefname{对象类型}{引用词汇}{引用词汇复数形式}
\Crefname{对象类型}{引用词汇}{引用词汇复数形式}
\end{Verbatim}

例如,可以使用以下代码将“定理”引用词汇修改为“命题”:
\begin{Verbatim}
\crefname{theorem}{命题}{命题}
\Crefname{theorem}{命题}{命题}
\end{Verbatim}
这样,使用 \verb|\cref| 命令引用定理对象时,就会自动生成“请参见命题 X”的效果。

需要同时引用几个并列的对象时,可以直接使用 \verb|\cref| 命令,例如:
\begin{Verbatim}
请参见\cref{sec:introduction,sec:conclusion}。
请参见\cref{sec:introduction,sec:math,sec:reference,sec:float}。
请参见\cref{sec:introduction,sec:math,sec:reference,,sec:float}。
请参见\cref{sec:introduction,sec:reference,sec:conclusion}。
请参见\cref{sec:introduction,sec:reference,sec:conclusion,%
  algo:algorithm,fig:parallel1,lst:fibonacci}。
\end{Verbatim}

这样,就可以在文档中产生“请参见章节 X 和 Y”的效果,其中 X 和 Y 为标签所在章节的编号:
\begin{itemize}
  \item 请参见\cref{sec:introduction,sec:conclusion}。
  \item 请参见\cref{sec:introduction,sec:math,sec:reference,sec:float}。
  \item 请参见\cref{sec:introduction,sec:math,sec:reference,,sec:float}。
  \item 请参见\cref{sec:introduction,sec:reference,sec:conclusion}。
  \item 请参见\cref{sec:introduction,sec:reference,sec:conclusion,%
        algo:algorithm,fig:parallel1,lst:fibonacci}。
\end{itemize}

上面的例子中,我们展示了如何使用 \verb|\cref| 命令引用多个并列的对象。在引用多个对象时,\pkg{cleveref} 宏包会自动根据对象类型生成正确的引用词汇,并使用适当的连接词将多个对象连接起来。

总之,交叉引用是撰写文档时非常有用的功能,可以方便读者查找相关内容。在使用交叉引用时,可以选择使用 \verb|\ref| 命令手动指定引用词汇,也可以使用 \verb|\cref| 命令自动根据对象类型生成引用词汇。同时,\pkg{cleveref} 宏包可以帮助用户自动设置正确的引用词汇,提高文档的可读性。
