\documentclass[UTF8,a4paper]{ctexart}

\usepackage{mathptmx}
\usepackage{amsmath,amsthm,amsfonts,amssymb,amscd}
\usepackage{multirow}
\usepackage{algorithm}
\usepackage{cases}
\usepackage{subfigure}
\usepackage{enumerate}
\usepackage{algorithmic}
\usepackage{extarrows}
\usepackage{latexsym}
\usepackage{longtable}
\usepackage{graphicx}
\usepackage{color}
\usepackage{etoolbox}
\usepackage{amsfonts}
\usepackage{caption}     %修改图片表格名称字体字号的宏包
\usepackage{float}        %自己设置图片插入位置
\usepackage{fancyhdr}     %设置页眉页脚的宏包
\setlength{\headheight}{43pt}
\usepackage{graphicx}     %插入图片
\usepackage{lastpage}     %引用最后一页
\usepackage{xcolor} 
\usepackage[colorlinks,linkcolor=black, citecolor=black, unicode={true}]{hyperref}
\graphicspath{{figures/}} %%图片路径
\usepackage[a4paper,top=3.2cm,bottom=3.8cm,left=3.3cm,right=1.8cm]{geometry}  %设置页边距



\usepackage{fancybox}                      %画边界线
\fancyput*(-0.8cm,-4.3cm){$|$}%
\fancyput*(-0.8cm,-4.9cm){$|$}%
\fancyput*(-0.8cm,-5.5cm){$|$}%
\fancyput*(-0.8cm,-6.1cm){$|$}%
\fancyput*(-0.8cm,-6.7cm){$|$}%
\fancyput*(-0.8cm,-7.3cm){$|$}%
\fancyput*(-0.8cm,-7.9cm){$|$}%
\fancyput*(-0.8cm,-8.5cm){$|$}%
\fancyput*(-0.8cm,-9.1cm){$|$}%
\fancyput*(-0.8cm,-9.7cm){$|$}%
\fancyput*(-1.0cm,-10.3cm){装}%
\fancyput*(-0.8cm,-10.9cm){$|$}%
\fancyput*(-0.8cm,-11.5cm){$|$}%
\fancyput*(-0.8cm,-12.1cm){$|$}%
\fancyput*(-0.8cm,-12.7cm){$|$}%
\fancyput*(-0.8cm,-13.3cm){$|$}%
\fancyput*(-1.0cm,-13.9cm){订}%
\fancyput*(-0.8cm,-14.5cm){$|$}%
\fancyput*(-0.8cm,-15.1cm){$|$}%
\fancyput*(-0.8cm,-15.7cm){$|$}%
\fancyput*(-0.8cm,-16.3cm){$|$}%
\fancyput*(-0.8cm,-16.9cm){$|$}%
\fancyput*(-1.0cm,-17.5cm){线}%
\fancyput*(-0.8cm,-18.1cm){$|$}%
\fancyput*(-0.8cm,-18.7cm){$|$}%
\fancyput*(-0.8cm,-19.3cm){$|$}%
\fancyput*(-0.8cm,-19.9cm){$|$}%
\fancyput*(-0.8cm,-20.5cm){$|$}%
\fancyput*(-0.8cm,-21.1cm){$|$}%
\fancyput*(-0.8cm,-21.7cm){$|$}%
\fancyput*(-0.8cm,-22.3cm){$|$}%
\fancyput*(-0.8cm,-22.9cm){$|$}%
\fancyput*(-0.8cm,-23.5cm){$|$}%

%设置目录格式(引文部分)
\usepackage[titles,subfigure]{tocloft}
\newcounter{algoline}
\newcommand\Numberline{\refstepcounter{algoline}\nlset{\thealgoline}}
\AtBeginEnvironment{algorithm}{\setcounter{algoline}{0}}
\makeatletter

\@addtoreset{equation}{section}

\makeatother
\renewcommand{\theequation}{\arabic{section}.\arabic{equation}}

\renewcommand{\cftdot}{$\cdot$}
\renewcommand{\cftdotsep}{1}
\setlength{\cftbeforesecskip}{10pt}
\setlength{\cftbeforesubsecskip}{3pt}
\setlength{\cftbeforesubsubsecskip}{0pt}
\renewcommand{\cftsecfont}{\zihao{5}\heiti}
\renewcommand{\cftsubsecfont}{\zihao{5}\heiti}
\renewcommand{\cftsubsubsecfont}{\zihao{5}\heiti}
\renewcommand{\cftsecleader}{\cftdotfill{\cftsecdotsep}}
\renewcommand{\cftsecdotsep}{\cftdotsep}
\renewcommand{\cftsecpagefont}{\zihao{5}}
\renewcommand{\cftsubsecpagefont}{\zihao{5}}
\renewcommand{\cftsubsubsecpagefont}{\zihao{5}}

%设置定理定义等的格式,如果某项的编号是跟着章节的,在后面加[section]
\theoremstyle{definition}
\newtheorem{thm}{定理\hspace{0.05pt}}[section]
\newtheorem{cor}{推论\hspace{0.05pt}}[section]
\newtheorem{lem}{引理\hspace{0.05pt}}[section]
\newtheorem{prop}{命题\hspace{0.05pt}}[section]
\newtheorem{conj}{猜想\hspace{0.05pt}}
\newtheorem{assume}{假设\hspace{0.05pt}}

\theoremstyle{definition}
\newtheorem{dfn}{定义\hspace{0.05pt}}[section]
\newtheorem{exmp}{例\hspace{0.05pt}}[section]
\newtheorem{rem}{注\hspace{0.05pt}}
\newtheorem*{pf}{证明}

%设置图表编号和章节相关联,参见http://blog.sina.com.cn/s/blog_5e16f1770100h6ts.html
\usepackage[T1]{fontenc}
%\usepackage[utf8]{inputenc}
%\usepackage[font=small,labelfont={bf,sf},tableposition=top]{caption}

\makeatletter
\renewcommand{\thefigure}{\ifnum \c@section>\z@ \thesection.\fi \@arabic\c@figure}
\renewcommand{\thetable}{\ifnum \c@section>\z@ \thesection.\fi \@arabic\c@table}
\makeatother

\newcommand{\primal}{\mathrm{primal}}
\newcommand{\dual}{\mathrm{dual}}
\newcommand{\trace}{\mathrm{tr}}
\newcommand{\vectorize}{\mathrm{vec}}
\newcommand{\prox}{\mathrm{prox}}
\newcommand{\mcL}{\mathcal{L}}
\newcommand{\st}{\mathrm{s.t.}}
\newcommand{\dom}{\mathrm{dom}}
\newcommand{\diag}{\mathrm{diag}}
\newcommand{\one}{\mathbf{1}}

\sloppy
\definecolor{lightgray}{gray}{0.5}

\makeatletter
\newenvironment{breakablealgorithm}
{% \begin{breakablealgorithm}
	\begin{center}
		\refstepcounter{algorithm}% New algorithm
		\hrule height.8pt depth0pt \kern2pt% \@fs@pre for \@fs@ruled
		\renewcommand{\caption}[2][\relax]{% Make a new \caption
			{\raggedright\textbf{算法~\thealgorithm} ##2\par}%
			\ifx\relax##1\relax % #1 is \relax
			\addcontentsline{loa}{algorithm}{\protect\numberline{\thealgorithm}##2}%
			\else % #1 is not \relax
			\addcontentsline{loa}{algorithm}{\protect\numberline{\thealgorithm}##1}%
			\fi
			\kern2pt\hrule\kern2pt
		}
	}{% \end{breakablealgorithm}
		\kern2pt\hrule\relax% \@fs@post for \@fs@ruled
	\end{center}
}
\makeatother

\renewcommand{\algorithmicrequire}{\textbf{输入: }}
\renewcommand{\algorithmicensure}{\textbf{输出: }}

\fancypagestyle{mainstyle}
{
   \fancyhf{}
   \pagenumbering{arabic}
   \fancyhead[L]{\qquad \includegraphics[height=1.34cm]{tongji.jpg}}   %页眉左侧插入同济大学logo
   \fancyhead[R]{\large 毕业设计(论文)报告纸~\\}
   \fancyfoot[R]{{\large 共\quad \pageref{LastPage}\quad 页\quad 第\quad \thepage \quad 页}} %偶数页左侧(LE),奇数页右侧(RO)标页码,oneside打印只用写RO
   \renewcommand{\headrulewidth}{1.5pt}  %页眉横线
   \renewcommand{\footrulewidth}{1.5pt}
}

\fancypagestyle{firststyle}
{
   \fancyhf{}
   \pagenumbering{Roman}
   \fancyhead[L]{\qquad \includegraphics[height=1.34cm]{tongji.jpg}}   %页眉左侧插入同济大学logo
   \fancyhead[R]{\large 毕业设计(论文)报告纸~\\}
   \fancyfoot[C]{\large \thepage}
   \renewcommand{\headrulewidth}{1.5pt}  %页眉横线
   \renewcommand{\footrulewidth}{0pt}
}
